\documentclass{report}

\usepackage{amsmath}
\usepackage{bytefield}
\usepackage{hyperref}

\begin{document}

\title{Apple Array System ABI}
\author{V. E. McHale}
\maketitle

\tableofcontents

\section{Arrays}

As an example, a {\tt 2x3} Apple array is laid out in memory like so:

\begin{bytefield}{10}
    \\
    \bitheader{0-8} \\
    \bitbox{1}{1} & \bitbox{2}{2 3} & \bitbox{3}{$a_{0i}$} & \bitbox{3}{$a_{1i}$}
\end{bytefield}

That is, rows are contiguous in memory.

Rank and dimensions are all 64-bit integers.

\section{Tuples}

Tuples are stack-allocated when it is not possible to pass them in registers. A tuple with an array as one of its elements is not flat but rather contains a pointer to the array.

\end{document}
