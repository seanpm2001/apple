\documentclass{article}

\usepackage[margin=1.5in]{geometry}
\usepackage{tikz}
\usepackage{hyperref}

\begin{document}

\title{Apple Array Allocation}
\author{V. E. McHale}
\maketitle

\begin{abstract}
    Array languages like J and APL suffer from a lack of embedability in implementations. Adroit memory management can make embedding easier; one would like to avoid thinking about ownership across two garbage collectors. Here I present statically determined memory allocation used in the Apple array system. Ownership is simple and Apple code does not constrain memory management in the host language.
\end{abstract}

\section{Introduction}

Array libraries like NumPy take inspiration from J, but our approach embeds a full language. Procedures can be called from a variety of languages (C, Python, Haskell) with the same type system---specialized for arrays---without compromising in order to accommodate Python's lack of static typing or C's lack of sophistication.

% Static memory management is likewise the same in all host languages.

% Moreover, an embedded language can perform deforestation or fusion, which is not possible with NumPy's library approach.

The language is expression-oriented (immutable); there are no references.
% Let us reflect on this unexpected :
% no case/sum types... etc.

\section{Method}

Our work is based on classical liveness algorithms; we annotate statements with {\tt uses} and {\tt defs} and thence compute liveness intervals for arrays. % the IR generation step only specifies allocations.

When generating the IR, we track each mention of the array with a label and associate the label with the temporary that can be used to free it.

% from stackexchange; don't know what it does
\tikzset{
block/.style = {draw, minimum height=2.5em, minimum width=4em, node distance=1.75cm}
}

\begin{tikzpicture}[auto]
    \node [block] (plain) {\tt [Stmt]};
    \node [block, below of=plain] (live) {\tt [(Stmt, Liveness)]};
    \node [block, below of=live] (alloc) {\tt [Stmt]};
    \draw [->] (plain) -- node {Liveness Analysis} (live);
    \draw [->] (live) -- node {Insert Frees} (alloc);
\end{tikzpicture}

% cite appel
%, though we must tag additional information when generating the IR.

The IR used in the Apple compiler is sequences of statements and expressions.

\begin{verbatim}
data ArrayExp = ArrayPointer Temp (Maybe Exp) (Maybe Int) -- register, offset, label

data Exp = Const Int
         | Reg Temp
         | At ArrayExp
         ...

data Stmt = Malloc Int Temp Exp -- label, register, size
          | Write ArrayExp Exp
          | MovTemp Temp Exp
          ...
\end{verbatim}

\subsection{Generation}

We must be careful when arranging loops;

\subsection{At Work}

As an example, suppose we wish to generate an array and extract the first element:

\begin{verbatim}
 > {. (irange 0 99 1)
0
\end{verbatim}

Under the hood:

\begin{verbatim}
 > :ir {. (irange 0 99 1)
(movtemp r_14 (int 0))
(movtemp r_15 (int 99))
(movtemp r_13 (+ (- (reg r_15) (reg r_14)) (int 1)))
(malloc r_12 : (+ (asl (reg r_13) (int 3)) (int 16)))
(write (ptr r_12) (int 1))
(write (ptr r_12+(int 8)) (reg r_13))
(movtemp r_16 (int 0))
(mjump (>= (reg r_16) (reg r_13)) apple_1)

apple_0:
(write (ptr r_12+(+ (asl (reg r_16) (int 3)) (int 16))) (reg r_14))
(movtemp r_14 (+ (reg r_14) (int 1)))
(movtemp r_16 (+ (reg r_16) (int 1)))
(mjump (< (reg r_16) (reg r_13)) apple_0)

apple_1:
(movtemp r_ret @(ptr r_12+(int 16)))
(free r_12)
\end{verbatim}

That is, the array that is allocated is freed precisely after all relevant operations have completed.

% note that the way we exit loops matters

\section{Coda}

Apple has the potential to be far more efficient; one could consolidate allocations, e.g.

\begin{verbatim}
 irange 0 10 1 ++ irange 0 10 1
\end{verbatim}

performs one allocation for each {\tt irange 0 10 1} but this could be

\end{document}
