%! TeX program = lualatex

\documentclass[sigplan,screen]{acmart}

\usepackage{bytefield}
\usepackage{tikz}
\usepackage{hyperref}
\usepackage{url}

\begin{document}

% https://mirror.math.princeton.edu/pub/CTAN/macros/unicodetex/latex/fontspec/fontspec.pdf
\setmonofont{Jet Brains Mono}[Scale=MatchAveragecase]

% emphasize embedability as motivation (lack of popularity of APL, shocking popularity of Python... futhark/accelerate)
% Then show that the method allows our own constructs hm

% export functions, not "calls into an array system"

\begin{abstract}
    Array languages like J and APL suffer from a lack of embedability in implementations. Adroit memory management can make embedding easier; one would like to avoid thinking about ownership across two garbage collectors and runtime linking is simpler. Here I present statically determined memory allocation used in the Apple array system, a JIT compiler. Ownership is simple and Apple code does not constrain memory management in the host language. This is followed by exhibition of two embeddings, one in Python and one in R.
\end{abstract}

\title{Apple Array Allocation}
\orcid{0000-0001-6093-2967}
\author{V. E. McHale}
\affiliation{%
\institution{Northern Trust}
  \streetaddress{50 South LaSalle Street}
  \city{Chicago}
  \state{IL}
  \postcode{IL}
  \country{USA}
}
\email{vamchale@gmail.com}
\maketitle

\section{Introduction}

% Kell paper
% Intro should point out lack of success in array languages...

% full language incl.
% array languages fail to export their constructs (trains)
Array libraries like NumPy take inspiration from J, but our approach embeds a full language. Procedures can be called from a variety of languages (C, Python, Haskell) with the same type system---specialized for arrays---without compromising in order to accommodate Python's lack of static typing or C's lack of sophistication. Moreover, an embedded compiler can perform deforestation and fusion, offering something over a shared library.

\cite{hui2020}

% cite Hui on reference counting

The language is expression-oriented (immutable); there are no references. This constraint, with extra bookkeeping information added during IR generation, is responsible for the surprising fact that memory allocation can be completely determined at compile time.

% Also show off embedding results

\section{Method}

Our work is based on established liveness algorithms; we annotate statements with {\tt uses} and {\tt defs} and thence compute liveness intervals for arrays.
% control flow

\tikzset{
block/.style = {draw, minimum height=2.5em, minimum width=4em, node distance=1.75cm}
}

\begin{tikzpicture}[auto]
    \node [] (expr) {\ldots};
    \node [block, below of=expr] (plain) {\tt [Stmt]};
    \node [block, below of=plain] (live) {\tt [(Stmt, Liveness)]};
    \node [block, below of=live] (alloc) {\tt [Stmt]};
    \draw [->] (expr) -- node {IR Generation} (plain);
    \draw [->] (plain) -- node {Liveness Analysis} (live);
    \draw [->] (live) -- node {Insert Frees} (alloc);
\end{tikzpicture}

% We insert {\tt free}s mechanically, % frees are calculated and thence less fickle (double free/leftover)
% allocations are specified but not frees...

The IR used in the Apple compiler is sequences of statements and expressions, viz.

\begin{verbatim}
data Exp = Const Int
         | Reg Temp
         | At ArrayPtr
         ...

data Stmt = MovTemp Temp Exp
          | Write ArrayPtr Exp
          | Malloc Lbl Temp Exp -- label, register, size
          | CondJump Exp Loc
          | Jump Loc | Label Loc
          ...
\end{verbatim}

Expressions can read from memory via {\tt At} and {\tt Stmt}s make use of memory access for writes, allocations etc. A function

\begin{verbatim}
aeval :: Expr -> IRM (Temp, Lbl, [Stmt])
\end{verbatim}
translates an array expression, assigning a {\tt Lbl} for subsequent access and associating the labeled array with a {\tt Temp} to be used to free it.

% When generating the IR, we track each mention of the array with a label and associate the label with the temporary that can be used to free it.
All accesses to an array use the {\tt ArrayPtr} type, which specifies the {\tt Lbl} for tracking within the compiler.

\begin{verbatim}
-- register, offset, label
data ArrayPtr = ArrayPtr Temp (Maybe Exp) Lbl
\end{verbatim}

Then one has

\begin{verbatim}
defs :: Stmt -> IntSet
defs (Malloc l _ e) = singleton l
defs _              = empty

uses :: Stmt -> IntSet
uses (Write (ArrayPtr _ _ l) e) = insert l (defsE e)
uses ...
\end{verbatim}

% https://www.cs.rice.edu/~kvp1/spring2008/lecture7.pdf

Note that with such labels one can access the array from different {\tt Temp}s; this turns out to be necessary for generating efficient code in the compiler.

% cite appel
% though we must tag additional information when generating the IR.
% Explain the liveness interval criterion (simply new/done)

% explicitly point out that we look up the Temp argument to free based on label (which is from live intervals)

\subsection{Generation}

% detail: live intervals, specify that
% This turns out to be okay

We must take care when arranging loops; the loop should exit by continuing rather than jumping to an exit location. Then we can insert {\tt free}s precisely at the end of the liveness interval without worrying whether the {\tt free} statement is reachable.

For instance, we do not write:

\begin{verbatim}
apple_0:
(condjump (< (reg r_2) (int 2)) apple_1)
(movtemp r_0
    @(ptr r_1+(+ (asl (reg r_2) (int 3)) (int 16))))
(jump apple_0)

apple_1:
\end{verbatim}

But rather:

\begin{verbatim}
(condjump (>= (reg r_2) (int 2)) apple_1)

apple_0:
(movtemp r_0
    @(ptr r_1+(+ (asl (reg r_2) (int 3)) (int 16))))
(condjump (< (reg r_2) (int 2) apple_0)

apple_1:
\end{verbatim}

One might ask why freeing at the end of a live interval guarantees that the {\tt free} statement is reachable.
% since our

\subsection{Flat Array Types}

Apple arrays are completely flat in memory, consisting of rank, dimensions, and data laid out contiguously.

A {\tt 2x4} array:

\begin{bytefield}[bitwidth=0.075\linewidth]{11}
    \\
    \bitheader{0-10} \\
    \bitbox{1}{2} & \bitbox{1}{2} & \bitbox{1}{4} & \bitbox{1}{$a_{00}$} & \bitbox{1}{$a_{01}$} & \bitbox{1}{$a_{02}$} & \bitbox{1}{$a_{03}$} & \bitbox{1}{$a_{10}$} & \bitbox{1}{$a_{11}$} & \bitbox{1}{$a_{12}$} & \bitbox{1}{$a_{13}$}
\end{bytefield}

This is not strictly necessary but it simplifies things for us; it means we only need one call to the system's {\tt free}.

All data are flat; there are only two primitive types: 64-bit integers and 64-bit floats. Arrays of tuples are supported but not arrays of tuples of arrays; there are no user-defined types.

This turns out to be enough % NumPy
% hope is bidirectional/fluent embeddings = it doesn't matter.

\subsection{At Work}

We can inspect the generated code in the REPL. The following generates an array and extracts the first element:

\begin{verbatim}
 > {. (irange 0 99 1)
0
\end{verbatim}

We can

\begin{verbatim}
 > :ir {. (irange 0 99 1)
(movtemp r_14 (int 0))
(movtemp r_15 (int 99))
(movtemp r_13 (+ (- (reg r_15) (reg r_14)) (int 1)))
(malloc r_12 : (+ (asl (reg r_13) (int 3)) (int 16)))
(write (ptr r_12) (int 1))
(write (ptr r_12+(int 8)) (reg r_13))
(movtemp r_16 (int 0))
(condjump (>= (reg r_16) (reg r_13)) apple_1)

apple_0:
(write
    (ptr r_12+(+ (asl (reg r_16) (int 3)) (int 16)))
    (reg r_14))
(movtemp r_14 (+ (reg r_14) (int 1)))
(movtemp r_16 (+ (reg r_16) (int 1)))
(condjump (< (reg r_16) (reg r_13)) apple_0)

apple_1:
(movtemp r_ret @(ptr r_12+(int 16)))
(free r_12)
\end{verbatim}

That is, the array allocated for {\tt irange 0 99 1} is freed precisely when it is no longer in use.

\section{Embeddings}

Apple has been embedded in Python and R. See the following interactions:

\begin{verbatim}
>>> import apple
>>> area=apple.cache('''
λas.λbs.
    { Σ ⇐ [(+)/x]
    ; 0.5*abs.(Σ ((*)`as (1⊖bs)) - Σ ((*)`(1⊖as) bs))
    }
''')
>>> import numpy as np
>>> apple.f(area,np.array([0,0,3.]),np.array([0,4,4.]))
6.0
\end{verbatim}

\begin{verbatim}
> source("./apple.R")
> shoelace<-cache("
λas.λbs.
    { Σ ⇐ [(+)/x]
    ; 0.5*abs.(Σ ((*)`as (1⊖bs)) - Σ ((*)`(1⊖as) bs))
    }
")
> run(shoelace,c(0,0,3),c(0,4,4))
[1] 6
\end{verbatim}

One sees that we can run Apple code on NumPy arrays and R vectors without frustrating the garbage collector. Typical APL implementations 

Crucially, we do not depend on the host language's garbage collector to manage Apple arrays---this means that there is less code to be written for each language embedding.
% crucial for efficient embedding that it not require hooking into each GC
% also offers something more than NumPy

\begin{verbatim}
 > {shoelace ← λas.λbs. {Σ ⇐ [(+)/x]; 0.5*abs.(Σ ((*)`as (1⊖bs)) - Σ ((*)`(1⊖as) bs))}; shoelace ⟨0,0,3⟩ ⟨0,4,4⟩}
6.0
\end{verbatim}

\section{Coda}

Apple has the potential to be far more efficient; one could consolidate allocations, e.g.

\begin{verbatim}
 irange 0 20 1 ++ irange 0 10 1
\end{verbatim}
performs one allocation for each {\tt irange} but this could be consolidated into one---{\tt irange 0 20 1} is inferred to have type {\tt Vec 20 int} in the existing compiler; with liveness and size information one could do something like linear register allocation in which arrays are assigned to memory

This is a further advantage of completely flat arrays

% slot diagrams?

\bibliographystyle{ACM-Reference-Format}
\bibliography{alloc.bib}

\end{document}
